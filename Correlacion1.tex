% Options for packages loaded elsewhere
\PassOptionsToPackage{unicode}{hyperref}
\PassOptionsToPackage{hyphens}{url}
%
\documentclass[
]{article}
\usepackage{amsmath,amssymb}
\usepackage{iftex}
\ifPDFTeX
  \usepackage[T1]{fontenc}
  \usepackage[utf8]{inputenc}
  \usepackage{textcomp} % provide euro and other symbols
\else % if luatex or xetex
  \usepackage{unicode-math} % this also loads fontspec
  \defaultfontfeatures{Scale=MatchLowercase}
  \defaultfontfeatures[\rmfamily]{Ligatures=TeX,Scale=1}
\fi
\usepackage{lmodern}
\ifPDFTeX\else
  % xetex/luatex font selection
\fi
% Use upquote if available, for straight quotes in verbatim environments
\IfFileExists{upquote.sty}{\usepackage{upquote}}{}
\IfFileExists{microtype.sty}{% use microtype if available
  \usepackage[]{microtype}
  \UseMicrotypeSet[protrusion]{basicmath} % disable protrusion for tt fonts
}{}
\makeatletter
\@ifundefined{KOMAClassName}{% if non-KOMA class
  \IfFileExists{parskip.sty}{%
    \usepackage{parskip}
  }{% else
    \setlength{\parindent}{0pt}
    \setlength{\parskip}{6pt plus 2pt minus 1pt}}
}{% if KOMA class
  \KOMAoptions{parskip=half}}
\makeatother
\usepackage{xcolor}
\usepackage[margin=1in]{geometry}
\usepackage{color}
\usepackage{fancyvrb}
\newcommand{\VerbBar}{|}
\newcommand{\VERB}{\Verb[commandchars=\\\{\}]}
\DefineVerbatimEnvironment{Highlighting}{Verbatim}{commandchars=\\\{\}}
% Add ',fontsize=\small' for more characters per line
\usepackage{framed}
\definecolor{shadecolor}{RGB}{248,248,248}
\newenvironment{Shaded}{\begin{snugshade}}{\end{snugshade}}
\newcommand{\AlertTok}[1]{\textcolor[rgb]{0.94,0.16,0.16}{#1}}
\newcommand{\AnnotationTok}[1]{\textcolor[rgb]{0.56,0.35,0.01}{\textbf{\textit{#1}}}}
\newcommand{\AttributeTok}[1]{\textcolor[rgb]{0.13,0.29,0.53}{#1}}
\newcommand{\BaseNTok}[1]{\textcolor[rgb]{0.00,0.00,0.81}{#1}}
\newcommand{\BuiltInTok}[1]{#1}
\newcommand{\CharTok}[1]{\textcolor[rgb]{0.31,0.60,0.02}{#1}}
\newcommand{\CommentTok}[1]{\textcolor[rgb]{0.56,0.35,0.01}{\textit{#1}}}
\newcommand{\CommentVarTok}[1]{\textcolor[rgb]{0.56,0.35,0.01}{\textbf{\textit{#1}}}}
\newcommand{\ConstantTok}[1]{\textcolor[rgb]{0.56,0.35,0.01}{#1}}
\newcommand{\ControlFlowTok}[1]{\textcolor[rgb]{0.13,0.29,0.53}{\textbf{#1}}}
\newcommand{\DataTypeTok}[1]{\textcolor[rgb]{0.13,0.29,0.53}{#1}}
\newcommand{\DecValTok}[1]{\textcolor[rgb]{0.00,0.00,0.81}{#1}}
\newcommand{\DocumentationTok}[1]{\textcolor[rgb]{0.56,0.35,0.01}{\textbf{\textit{#1}}}}
\newcommand{\ErrorTok}[1]{\textcolor[rgb]{0.64,0.00,0.00}{\textbf{#1}}}
\newcommand{\ExtensionTok}[1]{#1}
\newcommand{\FloatTok}[1]{\textcolor[rgb]{0.00,0.00,0.81}{#1}}
\newcommand{\FunctionTok}[1]{\textcolor[rgb]{0.13,0.29,0.53}{\textbf{#1}}}
\newcommand{\ImportTok}[1]{#1}
\newcommand{\InformationTok}[1]{\textcolor[rgb]{0.56,0.35,0.01}{\textbf{\textit{#1}}}}
\newcommand{\KeywordTok}[1]{\textcolor[rgb]{0.13,0.29,0.53}{\textbf{#1}}}
\newcommand{\NormalTok}[1]{#1}
\newcommand{\OperatorTok}[1]{\textcolor[rgb]{0.81,0.36,0.00}{\textbf{#1}}}
\newcommand{\OtherTok}[1]{\textcolor[rgb]{0.56,0.35,0.01}{#1}}
\newcommand{\PreprocessorTok}[1]{\textcolor[rgb]{0.56,0.35,0.01}{\textit{#1}}}
\newcommand{\RegionMarkerTok}[1]{#1}
\newcommand{\SpecialCharTok}[1]{\textcolor[rgb]{0.81,0.36,0.00}{\textbf{#1}}}
\newcommand{\SpecialStringTok}[1]{\textcolor[rgb]{0.31,0.60,0.02}{#1}}
\newcommand{\StringTok}[1]{\textcolor[rgb]{0.31,0.60,0.02}{#1}}
\newcommand{\VariableTok}[1]{\textcolor[rgb]{0.00,0.00,0.00}{#1}}
\newcommand{\VerbatimStringTok}[1]{\textcolor[rgb]{0.31,0.60,0.02}{#1}}
\newcommand{\WarningTok}[1]{\textcolor[rgb]{0.56,0.35,0.01}{\textbf{\textit{#1}}}}
\usepackage{longtable,booktabs,array}
\usepackage{calc} % for calculating minipage widths
% Correct order of tables after \paragraph or \subparagraph
\usepackage{etoolbox}
\makeatletter
\patchcmd\longtable{\par}{\if@noskipsec\mbox{}\fi\par}{}{}
\makeatother
% Allow footnotes in longtable head/foot
\IfFileExists{footnotehyper.sty}{\usepackage{footnotehyper}}{\usepackage{footnote}}
\makesavenoteenv{longtable}
\usepackage{graphicx}
\makeatletter
\def\maxwidth{\ifdim\Gin@nat@width>\linewidth\linewidth\else\Gin@nat@width\fi}
\def\maxheight{\ifdim\Gin@nat@height>\textheight\textheight\else\Gin@nat@height\fi}
\makeatother
% Scale images if necessary, so that they will not overflow the page
% margins by default, and it is still possible to overwrite the defaults
% using explicit options in \includegraphics[width, height, ...]{}
\setkeys{Gin}{width=\maxwidth,height=\maxheight,keepaspectratio}
% Set default figure placement to htbp
\makeatletter
\def\fps@figure{htbp}
\makeatother
\setlength{\emergencystretch}{3em} % prevent overfull lines
\providecommand{\tightlist}{%
  \setlength{\itemsep}{0pt}\setlength{\parskip}{0pt}}
\setcounter{secnumdepth}{-\maxdimen} % remove section numbering
\ifLuaTeX
  \usepackage{selnolig}  % disable illegal ligatures
\fi
\IfFileExists{bookmark.sty}{\usepackage{bookmark}}{\usepackage{hyperref}}
\IfFileExists{xurl.sty}{\usepackage{xurl}}{} % add URL line breaks if available
\urlstyle{same}
\hypersetup{
  pdftitle={Correlaciones1},
  pdfauthor={Natalia Jimenez Guillen},
  hidelinks,
  pdfcreator={LaTeX via pandoc}}

\title{Correlaciones1}
\author{Natalia Jimenez Guillen}
\date{2024-02-26}

\begin{document}
\maketitle

\#\#\#Ejercicio 1

\begin{Shaded}
\begin{Highlighting}[]
\FunctionTok{library}\NormalTok{(readxl)}
\NormalTok{data }\OtherTok{\textless{}{-}} \FunctionTok{as.data.frame}\NormalTok{(}\FunctionTok{read\_excel}\NormalTok{(}\StringTok{"C:/data.xlsx"}\NormalTok{))}
\NormalTok{data}
\end{Highlighting}
\end{Shaded}

\begin{verbatim}
##    longitud ancho grosor   peso
## 1      12.4   3.6  17.36  167.0
## 2      22.6   4.3  21.82  342.1
## 3      17.9   4.1  13.54  322.9
## 4      10.2  10.2  40.90  154.8
## 5      16.8   5.7  34.06  358.1
## 6      13.3   4.1  35.36  227.9
## 7      14.1   5.8 108.64  323.8
## 8      10.2   5.9 125.64  285.2
## 9      22.5   6.2  80.20  613.8
## 10     16.9   3.6  60.48  254.3
## 11     19.1   4.1 124.70  310.1
## 12     25.8   4.7 195.78  426.8
## 13     22.5   3.9 121.58  521.2
## 14     27.6  10.2  33.12  765.1
## 15     38.0  10.2  61.58 1217.2
## 16     72.4   6.4  38.48 2446.5
## 17     37.5   3.9 104.94  675.7
## 18     10.2   2.7  22.24   90.9
## 19     11.6   2.0  35.74   86.8
## 20     10.8   2.7  54.68  109.1
## 21     11.4   1.8 260.88   67.7
## 22     10.2   2.8  46.76  204.5
## 23     10.2   3.3   0.00  170.3
## 24     18.6   2.7   0.00  176.8
## 25     24.4   4.4   0.00  543.2
## 26     23.5   4.5   0.00  628.2
## 27     24.8   3.5   0.00  401.0
## 28     14.1   3.9   0.00  302.4
## 29     24.6   4.8   0.00  623.5
## 30     30.9   6.0   0.00  978.9
## 31     20.2   5.7   0.00  607.9
## 32     12.8   2.8   0.00  165.6
## 33     16.9   3.6   0.00  307.9
## 34     14.2   2.8   0.00  192.4
## 35     18.0   5.3   0.00  524.7
## 36     11.7   2.4   0.00  111.2
## 37     14.1   2.4   0.00  178.7
## 38     17.7   3.9   0.00  273.4
## 39     36.6   6.0   0.00 1304.4
## 40     12.3   5.4   0.00  233.8
\end{verbatim}

\begin{Shaded}
\begin{Highlighting}[]
\CommentTok{\#Funcion para agregar coeficientes de correlación}
\NormalTok{panel.cor }\OtherTok{=} \ControlFlowTok{function}\NormalTok{(x, y, }\AttributeTok{digits =} \DecValTok{2}\NormalTok{, }\AttributeTok{prefix =} \StringTok{""}\NormalTok{, cex.cor, ...) \{}
\NormalTok{  usr }\OtherTok{\textless{}{-}} \FunctionTok{par}\NormalTok{(}\StringTok{"usr"}\NormalTok{)}
  \FunctionTok{on.exit}\NormalTok{(}\FunctionTok{par}\NormalTok{(usr))}
  \FunctionTok{par}\NormalTok{(}\AttributeTok{usr =} \FunctionTok{c}\NormalTok{(}\DecValTok{0}\NormalTok{,}\DecValTok{1}\NormalTok{,}\DecValTok{0}\NormalTok{,}\DecValTok{1}\NormalTok{))}
\NormalTok{  Cor }\OtherTok{\textless{}{-}} \FunctionTok{abs}\NormalTok{(}\FunctionTok{cor}\NormalTok{(x, y))}
\NormalTok{  txt }\OtherTok{\textless{}{-}} \FunctionTok{paste0}\NormalTok{(prefix, }\FunctionTok{format}\NormalTok{(}\FunctionTok{c}\NormalTok{(Cor, }\FloatTok{0.123456789}\NormalTok{), }\AttributeTok{digits =}\NormalTok{ digits)[}\DecValTok{1}\NormalTok{])}
  \ControlFlowTok{if}\NormalTok{(}\FunctionTok{missing}\NormalTok{(cex.cor)) \{}
\NormalTok{    cex.cor }\OtherTok{\textless{}{-}} \FloatTok{0.4} \SpecialCharTok{/} \FunctionTok{strwidth}\NormalTok{(txt)}
\NormalTok{  \}}
  \FunctionTok{text}\NormalTok{(}\FloatTok{0.5}\NormalTok{, }\FloatTok{0.5}\NormalTok{, txt,}
       \AttributeTok{cex =} \DecValTok{1} \SpecialCharTok{+}\NormalTok{ cex.cor }\SpecialCharTok{*}\NormalTok{ Cor ) }\CommentTok{\#Escalar texto a nivel de correlación }
\NormalTok{\}}

\CommentTok{\#Dibujamos la matriz de correlacion }
\FunctionTok{pairs}\NormalTok{(data, }
      \AttributeTok{upper.panel =}\NormalTok{ panel.cor,}
      \AttributeTok{lower.panel =}\NormalTok{ panel.smooth)}
\end{Highlighting}
\end{Shaded}

\begin{verbatim}
## Warning in par(usr): argument 1 does not name a graphical parameter

## Warning in par(usr): argument 1 does not name a graphical parameter

## Warning in par(usr): argument 1 does not name a graphical parameter

## Warning in par(usr): argument 1 does not name a graphical parameter

## Warning in par(usr): argument 1 does not name a graphical parameter

## Warning in par(usr): argument 1 does not name a graphical parameter
\end{verbatim}

\includegraphics{Correlacion1_files/figure-latex/unnamed-chunk-2-1.pdf}

\begin{Shaded}
\begin{Highlighting}[]
\FunctionTok{cor.test}\NormalTok{(data}\SpecialCharTok{$}\NormalTok{longitud, data}\SpecialCharTok{$}\NormalTok{peso)}
\end{Highlighting}
\end{Shaded}

\begin{verbatim}
## 
##  Pearson's product-moment correlation
## 
## data:  data$longitud and data$peso
## t = 19.989, df = 38, p-value < 2.2e-16
## alternative hypothesis: true correlation is not equal to 0
## 95 percent confidence interval:
##  0.9170685 0.9764377
## sample estimates:
##       cor 
## 0.9555894
\end{verbatim}

\begin{Shaded}
\begin{Highlighting}[]
\FunctionTok{library}\NormalTok{(correlation)}
\NormalTok{resultados }\OtherTok{\textless{}{-}} \FunctionTok{correlation}\NormalTok{(data)}
\NormalTok{resultados}
\end{Highlighting}
\end{Shaded}

\begin{verbatim}
## # Correlation Matrix (pearson-method)
## 
## Parameter1 | Parameter2 |         r |        95% CI |     t(38) |         p
## ---------------------------------------------------------------------------
## longitud   |      ancho |      0.40 | [ 0.10, 0.63] |      2.71 | 0.040*   
## longitud   |     grosor |  4.68e-03 | [-0.31, 0.32] |      0.03 | > .999   
## longitud   |       peso |      0.96 | [ 0.92, 0.98] |     19.99 | < .001***
## ancho      |     grosor | -1.29e-03 | [-0.31, 0.31] | -7.98e-03 | > .999   
## ancho      |       peso |      0.51 | [ 0.23, 0.71] |      3.64 | 0.004**  
## grosor     |       peso |     -0.06 | [-0.36, 0.26] |     -0.36 | > .999   
## 
## p-value adjustment method: Holm (1979)
## Observations: 40
\end{verbatim}

\#\#\#Rpubs

\#\#\#Ejercicio IV

\begin{Shaded}
\begin{Highlighting}[]
\FunctionTok{library}\NormalTok{(ggpubr)}
\end{Highlighting}
\end{Shaded}

\begin{verbatim}
## Loading required package: ggplot2
\end{verbatim}

\begin{Shaded}
\begin{Highlighting}[]
\FunctionTok{ggscatter}\NormalTok{(data, }\AttributeTok{x=} \StringTok{"longitud"}\NormalTok{, }\AttributeTok{y =} \StringTok{"peso"}\NormalTok{,}
          \AttributeTok{add =} \StringTok{"reg.line"}\NormalTok{, }\AttributeTok{conf.int =} \ConstantTok{TRUE}\NormalTok{, }
          \AttributeTok{Cor.coef =} \ConstantTok{TRUE}\NormalTok{, }\AttributeTok{cor.method =} \StringTok{"perso"}\NormalTok{, }
          \AttributeTok{xlab =} \StringTok{"Longitud piezas (mm)"}\NormalTok{, }\AttributeTok{ylab =} \StringTok{"Peso piezas (mg)"}\NormalTok{)}
\end{Highlighting}
\end{Shaded}

\includegraphics{Correlacion1_files/figure-latex/unnamed-chunk-5-1.pdf}

\begin{Shaded}
\begin{Highlighting}[]
\FunctionTok{library}\NormalTok{(corrplot)}
\end{Highlighting}
\end{Shaded}

\begin{verbatim}
## corrplot 0.92 loaded
\end{verbatim}

\begin{Shaded}
\begin{Highlighting}[]
\FunctionTok{corrplot}\NormalTok{(}\FunctionTok{cor}\NormalTok{(data))}
\end{Highlighting}
\end{Shaded}

\includegraphics{Correlacion1_files/figure-latex/unnamed-chunk-6-1.pdf}

\begin{Shaded}
\begin{Highlighting}[]
\NormalTok{distancia }\OtherTok{\textless{}{-}} \FunctionTok{c}\NormalTok{ (}\FloatTok{1.1}\NormalTok{, }\FloatTok{100.2}\NormalTok{, }\FloatTok{90.3}\NormalTok{, }\FloatTok{5.4}\NormalTok{, }\FloatTok{57.5}\NormalTok{, }\FloatTok{6.6}\NormalTok{, }\FloatTok{34.7}\NormalTok{, }\FloatTok{65.8}\NormalTok{, }\FloatTok{57.9}\NormalTok{, }\FloatTok{86.1}\NormalTok{)}
\NormalTok{n\_piezas }\OtherTok{\textless{}{-}} \FunctionTok{c}\NormalTok{(}\DecValTok{110}\NormalTok{,}\DecValTok{2}\NormalTok{,}\DecValTok{6}\NormalTok{,}\DecValTok{98}\NormalTok{, }\DecValTok{40}\NormalTok{, }\DecValTok{94}\NormalTok{, }\FloatTok{31.}\NormalTok{,}\DecValTok{5}\NormalTok{,}\DecValTok{8}\NormalTok{,}\DecValTok{10}\NormalTok{)}
\NormalTok{dist\_ncuent }\OtherTok{\textless{}{-}} \FunctionTok{data.frame}\NormalTok{(distancia, n\_piezas)}
\NormalTok{knitr}\SpecialCharTok{::}\FunctionTok{kable}\NormalTok{(dist\_ncuent)}
\end{Highlighting}
\end{Shaded}

\begin{longtable}[]{@{}rr@{}}
\toprule\noalign{}
distancia & n\_piezas \\
\midrule\noalign{}
\endhead
\bottomrule\noalign{}
\endlastfoot
1.1 & 110 \\
100.2 & 2 \\
90.3 & 6 \\
5.4 & 98 \\
57.5 & 40 \\
6.6 & 94 \\
34.7 & 31 \\
65.8 & 5 \\
57.9 & 8 \\
86.1 & 10 \\
\end{longtable}

\#\#\#b

\begin{Shaded}
\begin{Highlighting}[]
\FunctionTok{cor.test}\NormalTok{(dist\_ncuent}\SpecialCharTok{$}\NormalTok{distancia, dist\_ncuent}\SpecialCharTok{$}\NormalTok{n\_piezas)}
\end{Highlighting}
\end{Shaded}

\begin{verbatim}
## 
##  Pearson's product-moment correlation
## 
## data:  dist_ncuent$distancia and dist_ncuent$n_piezas
## t = -6.8847, df = 8, p-value = 0.0001265
## alternative hypothesis: true correlation is not equal to 0
## 95 percent confidence interval:
##  -0.9824414 -0.7072588
## sample estimates:
##        cor 
## -0.9249824
\end{verbatim}

\#Fiabilidad de un 95\%. Hay una parte que puede ser aleatoria, para
comprobar eso se debe calcular con el valor P, si sale mas de 0,05 es
correcto si sale menos es aleatorio

\#El coeficiente de significacion es de -0.9249824

\#\#\#c - d

\begin{Shaded}
\begin{Highlighting}[]
\NormalTok{correlation}\SpecialCharTok{::}\FunctionTok{correlation}\NormalTok{(dist\_ncuent)}
\end{Highlighting}
\end{Shaded}

\begin{verbatim}
## # Correlation Matrix (pearson-method)
## 
## Parameter1 | Parameter2 |     r |         95% CI |  t(8) |         p
## --------------------------------------------------------------------
## distancia  |   n_piezas | -0.92 | [-0.98, -0.71] | -6.88 | < .001***
## 
## p-value adjustment method: Holm (1979)
## Observations: 10
\end{verbatim}

\#El resultado de la correlacion (o,92) indica la relacion lineal
inversa quasiperfecta ya que se encuentra proximo a -1. Los valores del
intervalo de confianza del 95\% muestran el intervalo 8de valores) para
el coeficiente de correlacion {[}`-0.98, -0.71{]}. Atendiendo a los
valores ``p'' podemos afirmar que la correlacion es significativa ya que
el P-value (`0.001) es \textbf{es inferior al nivel de significancia}
'0.05' (ya que el IC es `0.95', el nivel de signicancia es de '0.05)

\#\#\#f \#La relación es significativa porque el valor de P es de
.001***, inferior al nivel de significancia 0,05.

\end{document}
